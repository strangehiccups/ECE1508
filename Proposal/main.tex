\documentclass[11pt,a4paper,BCOR12mm, headexclude, footexclude, openright]{scrartcl} 
\usepackage{newpxtext}
\usepackage[british]{babel}
\usepackage[utf8]{inputenc}
\usepackage{listings}
\usepackage[T1]{fontenc}
\usepackage{href-ul}
%\usepackage[scaled=0.85]{beramono}            % load a nice TT font
%\lstset{basicstyle=\ttfamily, language=sh}  
\lstset{columns=fullflexible,basicstyle=\ttfamily}
\usepackage{fancyhdr}
\usepackage{lastpage}
\usepackage{ifthen}
\usepackage{amsmath,amsfonts,amsthm}
%\usepackage{sfmath}
%\usepackage{eulervm}
\usepackage{makecell}
\usepackage{booktabs}
\usepackage{sectsty}
\usepackage{bbm}
\usepackage{commands}
\usepackage{tikz,pgfplots}
\usepackage{tikz-cd}
\usetikzlibrary{patterns,intersections,matrix,fit,trees,positioning,chains,shapes.geometric,shapes,angles,quotes,pgfplots.fillbetween}
\usetikzlibrary{arrows,math}
\usepackage{tikz-cd}
\usepackage{tikz-3dplot}
\usetikzlibrary{arrows.meta,bending}
\usepackage{tikzsymbols}
\usepackage{tikzpeople}
\usetikzlibrary{calc}
\usetikzlibrary{chains,positioning,shapes.symbols,fadings,shadows,backgrounds}
\usetikzlibrary{decorations.pathmorphing, decorations.pathreplacing}
\usetikzlibrary{shapes.callouts}
\usetikzlibrary{shapes.arrows,shadings}
\usetikzlibrary{decorations.text}
\usepackage{algorithm,algorithmic}
%\KOMAoptions{optionenliste}
%\KOMAoptions{Option}{Werteliste}
\usepackage{listings}
\usepackage{inconsolata}
\lstset{%
	numbers=left,
	numberstyle=\tiny,
	basicstyle=\ttfamily\fontfamily{courier}\footnotesize,
}


\renewcommand{\arraystretch}{1.1}
\newcommand{\horrule}[1]{\rule{\linewidth}{#1}}
\setlength{\textheight}{250mm}
\setlength{\textwidth}{175mm}
\setlength{\hoffset}{-25mm}
\setlength{\voffset}{-15mm}
\allsectionsfont{\centering \normalfont\scshape}


\numberwithin{equation}{section} 
\numberwithin{figure}{section}
\numberwithin{table}{section} 


\newcommand{\ProjectTopic}{Topic}
\newcommand{\ProjectCat}{Open-ended}
\setlength\parindent{0pt}

\fancypagestyle{plain}
{%
	\renewcommand{\headrulewidth}{0pt}%
	\renewcommand{\footrulewidth}{0.5pt}
	\fancyhf{}%
	\fancyfoot[R]{\emph{\footnotesize Page \thepage\ of \pageref{LastPage}}}%
	\fancyfoot[L]{\emph{\footnotesize \ProjectTopic}}%
}
\pagestyle{plain}

\newcommand{\question}[3]{
	\hrule
	\hrule
	\vspace{3pt}
	Question #1: #2 \hfill #3 Points
	\vspace{3pt}
	\hrule
	\hrule
	\vspace{7pt}
}

\newcommand{\code}[1]{\colorbox{gray!20}{\texttt{#1}}}
\newcommand{\textbox}[2]{
	\begin{center}
		\fbox{
			\begin{minipage}{ {#1} em}
				{#2}
			\end{minipage}
		}
	\end{center} 
}
\begin{document}
	
	\titlehead
	{
		\horrule{.5pt}\\
		University of Toronto\\%
		Department of Electrical and Computer Engineering\\%
		ECE1508: \textbf{Applied Deep Learning}
		\hfill
		A. Bereyhi - Winter 2026
		\vspace{-1ex}\\
		\horrule{.5pt}\\
	}
	\subject{}
	\title{ }
	\subtitle{\normalfont \textit{Course Project}\vspace*{-2.5cm}}
	\date{}
	\maketitle
	
	\textbox{43}{\textbf{Code of Honor.} 
		All external resources used in the project, including research papers, open-source repositories, datasets, and any content or code generated using AI tools, e.g., ChatGPT, GitHub Copilot, Claude, Gemini, must be \textit{clearly cited} in the final submission. The final report must also include \textit{a clear breakdown of individual group member contributions.} Any lack of transparency in the use of external resources or in reporting group contributions will be considered academic dishonesty and will significantly impact the final evaluation.
	}
	
\begin{table}[H]
	\begin{tabular}{l l}
		\hline
		\textbf{Topic} & Topic \\
		\hline
	\end{tabular}
\end{table}

\paragraph*{Objective} 
Automatic Speech Recognition (ASR) % evaluate the trade-offs of various ASR techniques?

\paragraph*{Motivation}
Here goes the motivation.

\paragraph*{Requirements} The final submission should address the following requirements while the details can be freely decided by the group members.
\begin{enumerate}
	\item Item 1
	\item Item 2
\end{enumerate}

\paragraph*{Milestones}
\begin{enumerate}
    \item Overall ASR architecture
	\item Data preparation and augmentation
        % \begin{itemize}
        %     \item resample and convert all audio clips to have uniform sample rate, channels, and duration
        %     \item audio augmentation (time shift, pitch shift, etc)
        %     \item convert uniform audio clips into Mel Spectrograms (?)
        %     \item apply Mel Frequency Cepstral Coefficients (MFCC) to Mel Spectrograms (?)
        %     \item spectrogram augmentation (frequency mask, time mask, etc)
        % \end{itemize}
    \item Design \& implementation of ASR model
        % \begin{itemize}
        %     \item generate feature maps (ResNet)
        %     \item feature maps to features (LSTM)
        %     \item features to character probabilities (Linear layer + softmax) for each timestep
        %     \item align input and output sequences (CTC)
        % \end{itemize}
    \item Training \& testing of ASR model
    \item Evaluation of ASR model
        % \begin{itemize}
        %     \item accuracy: word error rate (WER)
        %     \item real time performance
        %     \item robustness to noise
        % \end{itemize}
\end{enumerate}
	
\paragraph*{Submission Guidelines} The main body of work is submitted through Git. In addition, each group submits a final paper and gives a presentation. In this respect, please follow these steps.
\begin{itemize}
	\item Each group must maintain a Git repository, e.g., GitHub or GitLab, for the project. By the time of final submission, the repository should have
	\begin{itemize}
		\item Well-documented codebase
		\item Clear \texttt{README.md} with setup and usage instructions
		\item A \texttt{requirements.txt} file listing all required packages or an \texttt{environment.yaml} file with a reproducible environment setup
		\item Demo script or notebook showing sample input-output
		\item \textit{If applicable,} a \texttt{/doc} folder with extended documentation
	\end{itemize}
	\item A final report (maximum \textit{5 pages}) must be submitted in a PDF format. The report should be written in the provided formal style, including an abstract, introduction, method, experiments, results, and conclusion.\\
	\textbf{Important:} Submissions that do not use template are considered \textit{incomplete.}
	\item A 5-minute presentation (maximum \textit{5 slides including the title slide}) is given on the internal seminar on Week 15, i.e., \textit{Dec 8 to Dec 12,} by the group. For presentation, any template can be used.
\end{itemize}

	
	\paragraph*{Final Notes} While planning for the milestones please consider the following points.
	\begin{enumerate}
		\item You are encouraged to explore innovative approaches to conditioning or generation as long as the core objectives are met.
		\item While computational resources are limited, carefully chosen datasets and training setups can make even diffusion models feasible. Trade-offs, e.g., resolution, training steps, are expected and should be justified.
		\item Teams are expected to manage their computing needs and are advised to perform early tests to estimate runtime and training feasibility. As graduate students, team members can use facilities provided by the university, e.g., ECE Facility. Teams are expected to inform themselves about the limitations of the available computing resources and design the model accordingly.
	\end{enumerate}
	
	\bibliographystyle{plain}
	\bibliography{ref.bib}
	
\end{document}